\documentclass[12pt, a4paper, oneside]{ctexart}
\usepackage{amsmath, amsthm, amssymb, appendix, bm, graphicx, hyperref, mathrsfs}
\CTEXsetup[format={\Large\bfseries}]{section}
\title{\textbf{人工智能导论开题报告}}
\author{毛嘉贤}
\date{\today}
\linespread{1.5}
\newtheorem{theorem}{定理}[section]
\newtheorem{definition}[theorem]{定义}
\newtheorem{lemma}[theorem]{引理}
\newtheorem{corollary}[theorem]{推论}
\newtheorem{example}[theorem]{例}
\newtheorem{proposition}[theorem]{命题}
\renewcommand{\abstractname}{\Large\textbf{摘要}}


\begin{document}

\maketitle

\setcounter{page}{1}
\maketitle
\pagestyle{plain}

\newpage
\setcounter{page}{2}
\pagenumbering{arabic}
\renewcommand\refname{}
\section*{一、选题背景}
垃圾,可以定义为失去利用价值、无法利用的废弃物品,也可以认为是错误放置的社会资源。人类自诞生以来就在不断产生垃圾。垃圾的产生总是无可避免,但是随着时代的发展,我们可以用不同的手段来处理垃圾。在公元200年,古罗马建立了第一个有记录在案的环卫部门,聘请专门人员在街道上清理垃圾,把垃圾收集到马车上带走处理。在唐代,对于倾倒垃圾的管理十分严格,对于随意倾倒垃圾者,处以刑罚,有关管理部门如果没有履行职责,将同样获罪,并受处罚。虽然古代对垃圾作了一定的处理,但并未彻底解决问题,年深日久,仍会影响城市的居住环境。工业革命使得人类的生产力水平以指数级提升,也使得垃圾数量飞速上涨。仅2020年一年,中国生活垃圾清运量为23511.7万吨,较2019年减少了694.50万吨,同比减少2.87$\%$。大量的垃圾危害了人们的身体健康,也侵蚀着人类的活动区域面积。世界上一些大城市被垃圾包围,甚至形成垃圾城。所以,如何处置垃圾在全世界范围都是一个棘手的问题。
\par事实上,垃圾也是一种资源,绝大部分都可回收再利用。多种垃圾放置在一起就是垃圾,如果分开处理就可以变废为宝。垃圾分类,一般是指按一定规定或标准将垃圾分类储存、投放和搬运,从而转变成公共资源的一系列活动的总称。
\par在垃圾处理方面,德国、日本等发达国家一直处于领先地位。面对垃圾,资源尽可能循环利用是他们的第一选择:在源头控制和分类回收利用后,\\首先采取的是堆肥$\left(\text{生化}\right)$技术,其次是焚烧技术,最后才是卫生填埋垃圾。
\par发达国家中,德国的垃圾分类开展的最好。在2016年,德国的垃圾回收利用率高达67.1\%,超过了美国、日本,后2者的利用率分别为35\%与20.3\%。在垃圾投放问题上,发达国家不断完善体制机制建设,形成全社会共同参与、齐抓共管的有效格局。在社会意识基本形成后,客观上需要实施更为多元有效的制度安排,进一步综合运用社会治理手段、经济手段、法律手段、行政手段等,调动每一个社会主体的积极性和主动性,共同推动生活垃圾分类管理不断提质增效。
\par我国在垃圾乱排问题上没有立法和设置奖惩制度,只能靠人们的自觉来完成,并且公民从小的教育宣传不够,更加剧了后期的垃圾分拣处理难度。
\par在不进行垃圾分类的情况下,我国的混合垃圾处理分为3种形式:卫生填埋的占据绝大部分比例,堆肥次之,还有极少比例的焚烧。
\par卫生填埋是国内处理垃圾最主要方式,将垃圾在专门的地点埋起来,让它们在相当长的时间内自然降解,这种做法需要占用大量的土地资源,并且污染土地和地下水资源。堆肥,是指利用自然界广泛存在的微生物,有控制地促进固体废物中可降解有机物转化为稳定的腐殖质的生物化学过程;堆肥看似达到了重新利用资源的效果,但也容易造成土壤板结,保水、保肥性能减退的缺陷。垃圾焚烧是将垃圾放在高温炉中加热,使其可燃物部分完全氧化,产生的热量可以用来发电、供暖等,不可燃的部分变为残渣固体排出,使得剩余垃圾的重量和体积大大降低;但是如果没有提前对垃圾分好类,垃圾中的有机物燃烧后的有毒气体会污染大气,含重金属的残渣也会影响环境,并且垃圾焚烧厂的建设成本很高,政府部门财政支出压力大。
\par由此可见,合理地分类垃圾是最终处理垃圾之前的关键步骤。在公民将垃圾按类别投放进不同的垃圾箱之后,垃圾被统一运送到垃圾处理厂。国内目前的垃圾处理大多依靠人工分拣,效率太低,不能满足海量垃圾处理需求。并且工作在这样的场所对于劳动者健康不利。随着科技的发展,自动化技术将人从繁重的手工业中解放出来,已经成为不可逆转的趋势。在处理垃圾过程中,确定垃圾位置和识别垃圾种类,是实现自动化分拣的前提,计算机视觉技术可以在此发挥重要作用。
\par计算机视觉,可以认为是计算机图形学的逆过程。它是用摄像机图像代替人眼的视觉输入,用机器的处理代替人类的大脑分析,从数字图像中提取出来信息特征,从而让机器具备像人一样的视觉理解能力。计算机视觉起源于20世纪50年代,Hubel与Wiesel(1959)的实验发现视觉的前置部分不是处理视觉整体,而是简单的边缘结构。基于他们的工作,Larry Roberts于1963年在他的博士论文中研究了如何从图像中获取边缘结构。David Marr(1970)得出重要结论,视觉不仅是从边缘形状开始的,而且是分不同阶段与层次展开的。在20世纪90年代,计算机视觉有很多重大成果,其中之一便是关于特征提取。SIFT算法,即尺度不变特征变换算法,该算法自1999年由David Lowe提出以后被广泛的应用于图像识别,图像检索,3D重建等CV的各种领域
\par Navneet Dalal在2005年提出方向梯度直方图(HOG)算法;Bay H等(2006)\\改进了SIFT并提出加速鲁棒特征算法;Lazebnik S等提出了空间金字塔匹配算法;Pedro等提出了可变性部件模型,将物体看做多个组件的拼接,在目标检测问题上取得了很好的成果,在目标检测公开数据集Visual Object Class 上连续获得2007-2009年3年的冠军。
\par传统的图像处理方法需要开发者手动提取特征,需要针对特定问题采取特定方法,受数据集影响较大,鲁棒性差。深度学习在2012年后正式取代传统方法,成为目前计算机视觉中的图像识别以及目标检测的主流方法。
\par用深度学习方法进行垃圾图像分类,国内基本属于刚起步阶段。由于垃圾的形状和颜色差异较大,不容易手动提取类别特征,算法的识别精度不高,速度也不能满足实际生产需求。本文的工作目标是将深度学习方法用于垃圾图像识别的领域,开发垃圾分类中准确率高,速度快的深度学习算法,以利于设备快速分拣垃圾,提高效率,并降低人工成本,所以,本课题有较大的理论意义和应用价值。
\par开发垃圾图像分类系统的具体技术路线是:验证几大主流图像识别算法在垃圾分类上的应用,得到它们有关的训练准确率,在此基础上结合多种模型的优点,综合优化一种模型所存在的缺陷,寻找方法提高训练集和测试集的准确率。
\newpage
\section*{二、相关研究综述}
图像识别中的传统机器学习方法,需要靠设计者依据先验知识从图像中手动提取特征,然后将特征而不是图片输入进算法,得到输出的分类结果:这需要人的先验知识,如果特征选择不当,则分类效果不佳。与此不同,深度学习不需要开发者手动提取特征,只需要将原始图像直接输入进神经网络,训练网络中的权值以及偏置即可。
\par斯坦福大学的Mindy Yang等人(2016)构建了公开的垃圾分类数据集\\TrashNet Dataset,包含6类垃圾共计2527张图像,其中玻璃501张,纸类594张,硬纸板403张,塑料482张,金属410张和一般垃圾137张。Mindy Yang在这一数据集上初步做了实验,用SVM方法取得了63\%的准确率。RahmiArda Aral等人(2018)使用DenseNet121和DenseNet169网络,将平均准确率提升至95\%:更进一步,Umut Ozkaya等(2019)对比了多种分类提取特征的网络与分类器的搭配,发现GoogleNet搭配SVM分类器是效果最好的,准确率达到了97.86\%,这是目前为止在TrashNet Dataset数据集上最好的效果
\par国内方面在垃圾分类领域的研究较少,齐鑫宇,李佳航(2021)针对目前图像局部特征表达存在的复杂性,模糊性等不足,采用特征多层池化以及系统神经网络学习的方式进行优化。同时在ResNet101模型的基础上设计并构建了基于CNN算法的新模型框架,此系统模型也能实现端与端的实时识别。新模型提高了对训练样本图像信息提取的精确度以及图片识别的准确率,实验表明准确率平均提高了10\%,为未来实现人工智能垃圾分类提供图像识别模型基础。
\par以上这些研究存在的主要问题包括:垃圾类别和图片数目太少,大多数图片的背景单一,算法的泛化性得不到验证。
\newpage
\section*{三、拟解决的问题和研究内容}
ResNet要解决的是深度神经网络的退化问题,即使用浅层直接堆叠成深层网络,不仅难以利用深层网络强大的特征提取能力,而且准确率会下降,这个退化不是由于过拟合引起的。
\par ResNet也称为残差网络。ResNet是由残差块(Residual Building Block)构建的,论文截图如下所示:提出了两种映射:identity mapping(恒等映射),指的是右侧标有x的曲线;residual mapping(残差映射),残差指的是F(x)部分。最后的输出是F(x)+x。F(x)+x的实现可通过具有``shortcut connections”的前馈神经网络来实现。shortcut connections是跳过一层或多层的连接。图中的``weight layer”指卷积操作。如果网络已经达到最优,继续加深网络,residual mapping将变为0,只剩下identity mapping,这样理论上网络会一直处于最优状态,网络的性能也就不会随着深度增加而降低。
\par 本文基于深度学习对垃圾分类算法进行研究,详细分析了深度学习在图像识别分类领域的应用。对现有的经典图像分类模型进行分析验证,设计实验对比各模型在本课题的适用性。本课题主要的工作内容如下:
\begin{itemize}
    \item 分析垃圾分类与深度学习的发展现状,并且对现有垃圾分类算法进行分析。
    \item 对于深度学习领域中历届经典的图像分类模型进行分析,当模型出现过拟合或者欠拟合问题时,了解常见解决过拟合欠拟合问题的方法。
    \item 构建垃圾分类数据集,将图像数据进行预处理并且使用图像增强方式增加模型的鲁棒性,分析迁移学习对本任务的适应性,测试各模型对垃圾分类任务的准确率。
    \item 分析现有模型可能存在的缺陷,寻找方法尝试提高训练集和测试集的准确率。
\end{itemize}
\newpage
\section*{四、可行性分析}
\subsection*{研究设计合理}
本课题研究小组在文献分析和网络调研基础上,充分了解中国垃圾分类问题的现状,在此基础上确定了``基于深度学习的垃圾图像分类识别系统"的课题主题,结合前文的国内外研究现状,本研究已经有大量的前期工作基础,因此课题设计合理。
\subsection*{前期工作扎实,科研资源丰富}
本课题研究小组根据Github上的开源代码已经积累相关经验,小组成员具备Python可视化编程能力,LaTeX撰写课程论文能力,能够正常开展研究工作。
\subsection*{分工科学,目标明确}
小组成员具体分工如下:
\begin{itemize}
    \item 毛嘉贤: 撰写报告,查找资料,阅读文献,学习代码框架。
    \item 林树敏: 撰写报告,配置环境,学习相关代码框架,运行并调试项目代码,训练模型并分析结果。
    \item 张家铭: 查找资料,学习项目相关配置和算法,配置环境,学习相关框架,撰写报告。
\end{itemize}
\subsection*{研究资源充分}
本课题研究小组拥有Github上大量有关图像识别算法的代码模型,且拥有庞大的垃圾分类图像集如华为垃圾分类图像等,能够很好开展训练任务。
\newpage
\section*{五、计划进度安排}
\begin{itemize}
    \item 第1阶段(2023-2-18-2023-4-3)-查找文献,确定选题,拟定具体的研究计划和路线,补充相关理论知识,撰写开题报告。
    \item 第2阶段(2023-4-3-2023-5-8)-收集相关数据集进行训练,收集相关算法模型进行研究,制定初版研究方案。
    \item 第3阶段(2023-5-8-2023-6-5)-根据训练集的结果进行算法优化,得到研究结果,完成项目后续报告的编写,进行结题。
\end{itemize}

\newpage
\begingroup
\section*{六、参考文献}
\renewcommand{\section}[2]{}
\begin{thebibliography}{99}
    \addtolength{\itemsep}{-1.5ex}
    \bibitem{1}陈伟.\emph{基于深度学习的垃圾分类算法研究}[D].天津职业技术师范大学,2021.DOI:10.27711/d.cnki.gtjgc.2021.000065.
    \bibitem{2}董子源.\emph{基于深度学习的垃圾分类系统设计与实现}[D].中国科学院大学(中国科学院沈阳计算技术研究所),2020.DOI:10.27587/d.cnki.gksjs.2020.000005.
    \bibitem{3}齐鑫宇.\emph{基于深度学习的垃圾图片处理与识别}[J].电脑知识与技术,2021,17(09):20-24.DOI:10.14004/j.cnki.ckt.2021.0819.
    \bibitem{4}腾讯云.\emph{如何构建用于垃圾分类的图像分类器}[EB/OL].(2019-06-21).https://cloud.tencent.com/developer/article/1449642
    \bibitem{5}trashnet.\emph{开源代码参考}[EB/OL].(2017-06-21).https://github.com/garythung/trashnet
\end{thebibliography}

\end{document}